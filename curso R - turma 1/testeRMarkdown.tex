% Options for packages loaded elsewhere
\PassOptionsToPackage{unicode}{hyperref}
\PassOptionsToPackage{hyphens}{url}
%
\documentclass[
]{article}
\usepackage{lmodern}
\usepackage{amssymb,amsmath}
\usepackage{ifxetex,ifluatex}
\ifnum 0\ifxetex 1\fi\ifluatex 1\fi=0 % if pdftex
  \usepackage[T1]{fontenc}
  \usepackage[utf8]{inputenc}
  \usepackage{textcomp} % provide euro and other symbols
\else % if luatex or xetex
  \usepackage{unicode-math}
  \defaultfontfeatures{Scale=MatchLowercase}
  \defaultfontfeatures[\rmfamily]{Ligatures=TeX,Scale=1}
\fi
% Use upquote if available, for straight quotes in verbatim environments
\IfFileExists{upquote.sty}{\usepackage{upquote}}{}
\IfFileExists{microtype.sty}{% use microtype if available
  \usepackage[]{microtype}
  \UseMicrotypeSet[protrusion]{basicmath} % disable protrusion for tt fonts
}{}
\makeatletter
\@ifundefined{KOMAClassName}{% if non-KOMA class
  \IfFileExists{parskip.sty}{%
    \usepackage{parskip}
  }{% else
    \setlength{\parindent}{0pt}
    \setlength{\parskip}{6pt plus 2pt minus 1pt}}
}{% if KOMA class
  \KOMAoptions{parskip=half}}
\makeatother
\usepackage{xcolor}
\IfFileExists{xurl.sty}{\usepackage{xurl}}{} % add URL line breaks if available
\IfFileExists{bookmark.sty}{\usepackage{bookmark}}{\usepackage{hyperref}}
\hypersetup{
  pdftitle={Natal 2020},
  pdfauthor={Jessica Ribas},
  hidelinks,
  pdfcreator={LaTeX via pandoc}}
\urlstyle{same} % disable monospaced font for URLs
\usepackage[margin=1in]{geometry}
\usepackage{graphicx,grffile}
\makeatletter
\def\maxwidth{\ifdim\Gin@nat@width>\linewidth\linewidth\else\Gin@nat@width\fi}
\def\maxheight{\ifdim\Gin@nat@height>\textheight\textheight\else\Gin@nat@height\fi}
\makeatother
% Scale images if necessary, so that they will not overflow the page
% margins by default, and it is still possible to overwrite the defaults
% using explicit options in \includegraphics[width, height, ...]{}
\setkeys{Gin}{width=\maxwidth,height=\maxheight,keepaspectratio}
% Set default figure placement to htbp
\makeatletter
\def\fps@figure{htbp}
\makeatother
\setlength{\emergencystretch}{3em} % prevent overfull lines
\providecommand{\tightlist}{%
  \setlength{\itemsep}{0pt}\setlength{\parskip}{0pt}}
\setcounter{secnumdepth}{-\maxdimen} % remove section numbering

\title{Natal 2020}
\author{Jessica Ribas}
\date{19/12/2020}

\begin{document}
\maketitle

\hypertarget{contando-da-minha-vida-para-vocuxeas}{%
\subsection{Contando da minha vida para
vocês}\label{contando-da-minha-vida-para-vocuxeas}}

Esse ano o Google está mais animado com as festas de fim de ano que
eu!!!

Desde o dia 01/12/20 já colocou os doodles de fim de ano e eu nem se
quer lembrava que já era dezembro, nem do meu aniversário estava
lembrabando!!!

\begin{figure}
\centering
\includegraphics{C:/Users/jessi/Documents/curso R/doodle.png}
\caption{Caption for the picture.}
\end{figure}

E o logo do R que funcionou pegando direto da internet sem precisar
salvar no computador!!!!

\begin{figure}
\centering
\includegraphics{http://developer.r-project.org/Logo/Rlogo-5.png}
\caption{Logo do R}
\end{figure}

\hypertarget{trabalhando-com-a-base-de-dados-iris}{%
\subsection{trabalhando com a base de dados
iris}\label{trabalhando-com-a-base-de-dados-iris}}

sei que tá um porre ver esses mesmos dados toda hora, eu também achava
isso quando estava aprendendo, mas agora gosto tando que não consigo
evitar usar ela!!!

\begin{verbatim}
##   Sepal.Length    Sepal.Width     Petal.Length    Petal.Width   
##  Min.   :4.300   Min.   :2.000   Min.   :1.000   Min.   :0.100  
##  1st Qu.:5.100   1st Qu.:2.800   1st Qu.:1.600   1st Qu.:0.300  
##  Median :5.800   Median :3.000   Median :4.350   Median :1.300  
##  Mean   :5.843   Mean   :3.057   Mean   :3.758   Mean   :1.199  
##  3rd Qu.:6.400   3rd Qu.:3.300   3rd Qu.:5.100   3rd Qu.:1.800  
##  Max.   :7.900   Max.   :4.400   Max.   :6.900   Max.   :2.500  
##        Species  
##  setosa    :50  
##  versicolor:50  
##  virginica :50  
##                 
##                 
## 
\end{verbatim}

\hypertarget{gruxe1fico-sobre-a-base-de-dados-iris}{%
\subsection{Gráfico sobre a base de dados
Iris}\label{gruxe1fico-sobre-a-base-de-dados-iris}}

esse gráfico foi feito de qualquer jeito, porque eu estou com preguiça
de pensar em como fazer graficos bonitos hoje, porém quem quiser pode
fazer um grafico bem feito usando a Rbase ou o ggplot2!!!!!

\includegraphics{testeRMarkdown_files/figure-latex/iris plot-1.pdf}

\end{document}
